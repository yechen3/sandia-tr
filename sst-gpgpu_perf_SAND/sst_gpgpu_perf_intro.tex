As the architectures of high-performance computing (HPC) evolves, there is a growing
need to understand and quantify the performance and design benefits of
emerging technologies. To complicate the design space, the rise of
General-Purpose Graphics Processing Units (GPGPUs) and other compute
accelerators, which are needed to handle the growing demands of compute-heavy
workloads, have become a necessary component in both high-performance
supercomputers and datacenter-scale systems. That the first exascale machines
will leverage the massively parallel compute capabilities of GPUs
\cite{snl_roadmap, ornl_roadmap, llnl_roadmap} is indicative of the growing
necessity of accelerator-based node architectures to obtain high compute
throughputs. As the software stack and
programming model of GPUs and their peer accelerators continue to improve, there
is every indication that this trend of accelerator integration will continue,
leading to a diverse ecosystem of technologies. GPUs are likely to continue to
play a role as discrete accelerators or integrated as a part of an SOC. As a
result, architects who wish to study the design of large-scale systems will need
to evaluate system and software designs using a GPU model. However, the focus of
all publicly available cycle-level simulators ({\em e.g.} GPGPU-Sim~\cite{gpgpu_sim})
to date has been on single-node performance. In order to truly study the problem at scale,
and to permit larger workloads to be evaluated, a parallelizable, multi-node GPU
simulator is necessary.

The Structural Simulation Toolkit (SST)~\cite{sst} is a parallel discrete
event-driven simulation framework that provides an infrastructure capable of
modeling a variety of high performance computing systems at many different
scales. Currently used by a wide variety of government agencies and computer
manufacturers to design and simulate HPC architectures, and, supported by a
Python and C++ code base with a large array of customization options, SST offers
the HPC community powerful, highly customizable, tools to create and integrate
models for evaluating current and future HPC node architectures and interconnect
networks. What has been lacking, up to this point, has been a method
to integrate accelerators into a node model in SST. This report builds upon
previous work~\cite{sst_gpgpu}, providing more details on our efforts to
integrate an open-source GPGPU simulator, GPGPU-Sim, into SST. This integration effort
will provide SST users the ability to run GPGPU-based simulations using the Balar
GPU component and will serve as a model for future accelerator integration
studies.

